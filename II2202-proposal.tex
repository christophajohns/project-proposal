\documentclass[12pt,twoside,english]{article}
\usepackage[utf8]{inputenc}

%%%%%%%%%%%%%%%%%%%%%%%%%%%%%%%%%%%%%%%%%%%%%%%%%%%%%%%%%%%%%%%%%%%%%%%
%% template for II2202 proposal
%% original 2020.08.28
%% revised  
%%%%%%%%%%%%%%%%%%%%%%%%%%%%%%%%%%%%%%%%%%%%%%%%%%%%%%%%%%%%%%%%%%%%%%%
%

%%% Local Variables:
%%% mode: latex
%%% TeX-master: "."
%%% End:

%%TC:ignore
\usepackage[paper=a4paper,dvips,top=1.5cm,left=1.5cm,right=1.5cm,
    foot=1cm,bottom=1.5cm]{geometry}


\usepackage{todonotes}          %to provide inline and margin notes
%\usepackage[T1]{fontenc}
%%\usepackage{pslatex}
\renewcommand{\rmdefault}{ptm} 
\usepackage{mathptmx}
\usepackage[scaled=.90]{helvet}
\usepackage{courier}
%
\usepackage{bookmark}

\usepackage{fancyhdr}
\pagestyle{fancy}

%%----------------------------------------------------------------------------
%%   pcap2tex stuff
%%----------------------------------------------------------------------------
 %%\usepackage[dvipsnames*,svgnames]{xcolor} %% For extended colors
 %%\usepackage{tikz}  %% Already loaded
 %%\usetikzlibrary{arrows,decorations.pathmorphing,backgrounds,fit,positioning,calc,shapes}

%% \usepackage{pgfmath}	% --math engine
%%----------------------------------------------------------------------------
%% \usepackage[latin1]{inputenc}
\usepackage[utf8]{inputenc} % inputenc allows the user to input accented characters directly from the keyboard
\usepackage[english]{babel}
%% \usepackage{rotating}		 %% For text rotating
\usepackage{array}			 %% For table wrapping
\usepackage{graphicx}	                 %% Support for images
\usepackage{float}			 %% Suppor for more flexible floating box positioning
\usepackage{color}                       %% Support for colour 
\usepackage{mdwlist}
%% \usepackage{setspace}                 %% For fine-grained control over line spacing
%% \usepackage{listings}		 %% For source code listing
%% \usepackage{bytefield}                %% For packet drawings
\usepackage{tabularx}		         %% For simple table stretching
%%\usepackage{multirow}	                 %% Support for multirow colums in tables
\usepackage{dcolumn}	                 %% Support for decimal point alignment in tables
\usepackage{url}	                 %% Support for breaking URLs
\usepackage[perpage,para,symbol]{footmisc} %% use symbols to ``number'' footnotes and reset which symbol is used first on each page
\usepackage[binary-units=true]{siunitx} %% to be able to use binary units
\newcommand{\SIadj}[2]{\SI[number-unit-product={\text{-}}]{#1}{#2}}

%% \usepackage{pygmentize}           %% required to use minted -- see python-pygments - Pygments is a Syntax Highlighting Package written in Python
%% \usepackage{minted}		     %% For source code highlighting

%% \usepackage{hyperref}		
\usepackage[all]{hypcap}	 %% Prevents an issue related to hyperref and caption linking
%% setup hyperref to use the darkblue color on links
%% \hypersetup{colorlinks,breaklinks,
%%             linkcolor=darkblue,urlcolor=darkblue,
%%             anchorcolor=darkblue,citecolor=darkblue}

%% Some definitions of used colors
\definecolor{darkblue}{rgb}{0.0,0.0,0.3} %% define a color called darkblue
\definecolor{darkred}{rgb}{0.4,0.0,0.0}
\definecolor{red}{rgb}{0.7,0.0,0.0}
\definecolor{lightgrey}{rgb}{0.8,0.8,0.8} 
\definecolor{grey}{rgb}{0.6,0.6,0.6}
\definecolor{darkgrey}{rgb}{0.4,0.4,0.4}
%% Reduce hyphenation as much as possible
\hyphenpenalty=15000 
\tolerance=1000

%% useful redefinitions to use with tables
\newcommand{\rr}{\raggedright} %% raggedright command redefinition
\newcommand{\rl}{\raggedleft} %% raggedleft command redefinition
\newcommand{\tn}{\tabularnewline} %% tabularnewline command redefinition

%% definition of new command for bytefield package
\newcommand{\colorbitbox}[3]{%
	\rlap{\bitbox{#2}{\color{#1}\rule{\width}{\height}}}%
	\bitbox{#2}{#3}}

%% command to ease switching to red color text
\newcommand{\red}{\color{red}}
%%redefinition of paragraph command to insert a breakline after it
\makeatletter
\renewcommand\paragraph{\@startsection{paragraph}{4}{\z@}%
  {-3.25ex\@plus -1ex \@minus -.2ex}%
  {1.5ex \@plus .2ex}%
  {\normalfont\normalsize\bfseries}}
\makeatother

%%redefinition of subparagraph command to insert a breakline after it
\makeatletter
\renewcommand\subparagraph{\@startsection{subparagraph}{5}{\z@}%
  {-3.25ex\@plus -1ex \@minus -.2ex}%
  {1.5ex \@plus .2ex}%
  {\normalfont\normalsize\bfseries}}
\makeatother

\setcounter{tocdepth}{3}	%% 3 depth levels in TOC
\setcounter{secnumdepth}{5}
%% Acronyms
\usepackage[acronym, nopostdot]{glossaries}
\glsdisablehyper
\makeglossaries

\renewcommand{\headrulewidth}{0pt}
%%%%%%%%%%%%%%%%%%%%%%%%%%%%%%%%%%%%%%%%%%%%%%%%%%%%%%%%%%%%%%%%%%%%
%% End of preamble
%%%%%%%%%%%%%%%%%%%%%%%%%%%%%%%%%%%%%%%%%%%%%%%%%%%%%%%%%%%%%%%%%%%%

\newacronym{ACK}{ACK}{Acknowledgement}
\newacronym{NACK}{NACK}{Negative Acknowledgement}
\newacronym{UDP}{UDP}{User Datagram Protocol}


\title{Proposing a Method for Investigating a Possible Tradeoff Between Enhancing Immersion and Impacting Energy Consumption by Enabling Automatic Naturalistic Lighting in Augmented Reality Games}
\author{
        \textsc{Stefano Formicola}
            \qquad
        \textsc{Christoph Albert Johns}
        \mbox{}\\
        \normalsize
            \texttt{formico}
        \textbar{}
            \texttt{cajohns}
        \normalsize
            \texttt{@kth.se}
}
\date{\today}


\lhead{II2202, Fall 2020, Period 1-2}
%% or \lhead{II2202, Fall 2020, Period 1}
\chead{Project proposal}
\rhead{\date{\today}}

\makeatletter
\let\ps@plain\ps@fancy 
\makeatother

\setlength{\headheight}{15pt}
\begin{document}

\maketitle


% \begin{abstract}
% \label{sec:abstract}

% Your abstract here.

% \end{abstract}
%%\clearpage

\selectlanguage{english}

\section{Allocation of responsibilities}
\label{sect:alloc_responsibilities}

Stefano Formicola is responsible for collecting data during the experiment, writing the sections regarding research questions, method, results, and discussion, and presenting the final project.
Christoph Albert Johns is responsible for creating the test application, writing the abstract and sections regarding introduction and literature study, and presenting the project plan.


\section{Organization}
\label{sect:organization}

The research project will be conducted as a two-person project on the basis of previously published work in the domain of virtual lighting in \gls{AR} scenes and user-based studies.
First, the project's theoretical foundation will be studied in parallel with designing and preparing the test application.
Second, the experiment setup will be decided and prepared. In parallel the test application will be programmed and the basic project presentation will be prepared.
Third, the study participants will be continuously recruited and the required data will be collected and analyzed in parallel.
Fourth, another literature study will be conducted to relate the project's findings to existing work.
Finally, the project presentation will be completed and presented.


\section{Background}
\label{sect:background}
The proposed research project builds on two main lines of research: prior works evaluating the relationship between visual characteristics of \gls{AR} objects and user experience (e.g. \cite{gabbard_effects_2006}) and research on the approximation of natural lighting conditions within virtual scenes in augmented reality (e.g. \cite{aittala_inverse_2010}).
While there has been extensive research on the effect of design decisions on several aspects of user experience such as immersion in digital games \cite{johnson_validation_2018} -- even in \gls{AR} games in a few cases (e.g. \cite{georgiou_development_2017}) -- the effect of lighting and more specifically automatic naturalistic lighting in \gls{AR} games seems not to have been examined yet.
To achieve naturalistic lighting, camera-sensor data is used to estimate and approximate the environment's lighting conditions and dynamically added the appropriate shading to the objects in a scene \cite{apple_arlightestimate_nodate,apple_pointlight_nodate}.
Depending on the scene and lighting conditions, this can be a computationally intensive process \cite{steed_constructing_2016}.
With automatic naturalistic lighting becoming increasingly available and common, e.g. through Apple's RealityKit API, it remains unclear whether this feature should be deployed by developers to enhance immersion -- especially since it could strongly affect energy consumption.

While several aspects of the experience when playing videos games in general and \gls{AR} games in particular have been defined and measured as well (see \cite{dey_systematic_2018, dunser_survey_2008} for overviews), we chose to investigate the effect of automatic naturalistic lighting on immersion because -- in contrast to, for example, \textit{flow} \cite{csikszentmihalyi_flow_1990} -- it is independent from optimal experience \cite{jennett_measuring_2008}, which our method will not be able to produce, and very likely to relate to visual presentation and, therefore, lighting \cite{jennett_measuring_2008}.
As Witmer and Singer argue, "scene realism" is one of the governing factors for presence in virtual environments \cite{witmer_measuring_1998}, which in turn affects how immersive a virtual environment appears \cite{jennett_measuring_2008}.
We argue that automatic naturalistic lighting enhances this sense of realism and will therefore affect user's sense of immersion.

\section{Problem statement}
\label{sect:problem_statement}

The project will investigate whether there is a tradeoff between an increased energy impact of possibly computationally intensive environment light probing and estimation and higher levels of immersion in augmented reality scenes.
To achieve naturalistic virtual lighting, video from the AR system's camera feed is continuously probed at a high frequency to estimate the environment lighting conditions.
These estimates are required input data for the used shading algorithms.
While there have been user-based studies evaluating visual characteristics of \gls{AR} objects, this particular aspect appears not to have been examined yet.

\section{Problem}
\label{sect:problem}

If there is a tradeoff between immersion and energy consumption, avoiding continuous \gls{AR} light estimation could free resources for either alternative uses or extended battery life.
If we find no tradeoff, developers and designers can be responsibly encouraged to implement features that enhance immersion.
Energy consumption is measured in terms of \gls{CPU} usage as there is no dedicated \gls{GPU} in the mobile devices we will be testing, so the \gls{CPU} will handle all processing related to the \gls{AR} experience, and because CPU usage can very finely be measured on the test device in contrast to general energy consumption.
That is why we will explore the following questions:

\begin{description}
    \item[RQ1.] Do users perceive a difference in immersion between enabled and disabled automatic naturalistic lighting?
    \item[RQ1a.] If they perceive a difference, are the participants able to tell that lighting is the cause?
    \item[RQ2.] How does enabling automatic naturalistic lighting affect \gls{CPU} usage?
\end{description}

\section{Hypothesis}
\label{sect:hypothesis}

While we are aware that we will only be able to test the proposed method on a small scale, we hypothesize the following outcomes for a full-scale experiment:

\begin{description}
    \item[H1.] Users experience higher immersion levels when automatic naturalistic lighting is enabled.
    \item[H1a.] Users are able to tell that the lighting was different between both runs of the experiment.
    \item[H2.] \gls{CPU} usage is higher when automatic naturalistic lighting is enabled.
\end{description}



\section{Purpose}
\label{sect:purpose}

The purpose of this project is to illuminate the relationship between immersion and energy impact in deploying advanced \gls{AR} features, thus empowering developers and designers to decide the most appropriate feature sets for their applications considering both user experience and sustainability aspects.

\section{Goal(s)}
\label{sect:goals}

The goal of this project is to propose and pre-test a method for exploring the relationship between immersion and energy impact when deploying automatic naturalistic lighting using a simple example application and an existing \gls{AR} \gls{API}.

\section{Tasks}
\label{sect:tasks}
\todo[inline]{Describe the tasks and sub tasks that are necessary to complete the work. Grouped into a work breakdown structure. For example:}

A simple multicast file server and a corresponding client will be implemented. A number of instances of the client will be executed with different starting times, but starting within \SI{30}{\second} of each other. Our test file will be a file of sufficient size to require more than \SI{30}{\second} to transmit when transported using UDP datagrams via a \SI{10}{Mbps} wired Ethernet. This file will contain a series of pseudo-random bytes. The total number of datagrams transmitted by all of the nodes will be compared when using positive ACK and when only generating NACK. The number of clients will vary from 1 to 1025, in increments of 8 clients.

\section{Method}
\label{sect:method}

The project will use an empirical approach to test two different lighting options in an experiment setting and analyze the results using statistical methods.
More specifically, we will modify an open-source \gls{AR} card-matching game \cite{cobb_maxxfrazerrealitykit-cardflip_2020} to include an option to enable and disable automatic naturalistic lighting using RealityKit's ".disableAREnvironmentLighting" render option.
The sample game will be presented to participants on an Apple iPad in an indoor setting with controlled lighting conditions where participants will be asked to complete one round of the card-matching game for each of the two render options.
The order of render options will be randomized to improve reliability.
The purpose of the experiment will be stated to the participants only in general terms as "investigating immersion in \gls{AR} games" to avoid influencing their perception of the scene by directing their attention towards lighting.
Participants will not be informed about the currently active render option for the same reason.
The sample application has been chosen due to its simplicity in functionality and visual presentation as well as general familiarity, thereby reducing the risk of errors in use or unwanted effects on immersion due to scene complexity.
To further negate this risk, participants will be given a trial with a third lighting condition: disabled automatic naturalistic lighting and an added point light above the scene.
The first run will be started, after the participant indicates that he is ready for the experiment to begin.

During each run, the IDE Xcode and more specifically its "Instruments" feature \cite{apple_xcode_nodate} will be used to log the test device's \gls{CPU} usage according to Apple's type definitions \cite{apple_system_nodate} in three second intervals.
While there is a more general power metric that can be logged through the same application \cite{apple_energy_nodate-1}, this metric is discrete and very inelastic and was, therefore, not used.
Additionally, the time to completion will be recorded from the moment the participants are handed the test device until the success message appears on screen.
We discussed and decided not to have participants think aloud during their app usage to avoid unwanted effects on their experience and, thus, immersion \cite{van_den_haak_retrospective_2003}.
After each run, participants will be asked to fill out the \gls{IEQ} (see \ref{sect:ieq}) to measure their levels of immersion dependent on the lighting treatment.
After the second run, participants will additionally be asked about their general experience using the application and elaborate on any differences they noticed between both runs.
While alternatives such as tracking eye movement or measuring reaction times could also be used to measure immersion \cite{jennett_measuring_2008}, we chose a questionnaire approach as the most economical option that still has satisfactory precedence in the literature \cite{boyle_engagement_2012}.
If a participant mentions one of the words "light", "lighting", "color", or "contrast" while answering these questions, the difference in lighting is considered as noticed by the user.
The expected duration of the experiment per participant is 30 minutes.
The expected number of participants in the small-scale pre-study of this project is ten.

After the above data has been successfully gathered, we will conduct a statistical analysis regarding the participants' immersion levels and the CPU usage between both lighting conditions using Student's two-sample \textit{t}-test to check for significant differences.
We chose this measure as it can be considered quite robust even with small and extremely small ($ N \leq 5 $) sample sizes \cite{de_winter_using_2013}.
If, however, the \textit{t}-test's assumption of normal distribution is not met, we will instead conduct a Mann-Whitney U-test to compare the two conditions following for example Nordin et al. \cite{nordin_attention_2013}.
Additionally, we will report the results of the interview after the experiment.
We are aware, however, that both the results of our analysis of variance as well as the results of our interviews should not be used to infer knowledge about either the effect of automatic naturalistic lighting on immersion or on energy consumption.
Rather, the analysis is carried out in order to evaluate the chosen methods and to illustrate how such analysis should be done if the same study were carried out on a larger scale.

\section{Milestone chart (time schedule)}
\label{sect:milestones}
\todo[inline]{Give a detailed schedule of how the project will be carried out. What is the project timeline and when will particularly meaningful points, referred to as milestones, be completed? What is the deliverable for each of these milestones? For example:}

The project will start on 25 August and end at 00:00 on 30 August. There will be the following milestones and deliverables:

\begin{description}
\item{26 August} working multicast server and client

\item{27 August} prepare testbed and perform experimental runs for each of the
  different numbers of clients to be completed (as per above) and stored in
  two comma separated files (ACK.csv and NACK.csv) which each line consisting
  of: number of clients, number of packets. Data will be collected using a
  separate computer running Wireshark. Analysis of the data and development of
  the analytic model will proceed in parallel with the data collection.

\item{28 August} model completed based upon using the statistical package R to do the curve fitting to the experimental data

\item{29 August} evaluation of model for additional data points to be check by new experimental runs to check that the model accurately produces the expected number of packets for the given number of clients for both ACK and NACK alternatives.

\item{Before 30 August} submit final report (the report will have been written in parallel with each of the above steps)
\end{description}

\bibliography{II2202-proposal}
% \bibliographystyle{IEEEtran}
\bibliographystyle{myIEEEtran}

\appendix
\section{Appendix}
\label{sect:appendix}

\subsection{Immersive Experience Questionnaire}
\label{sect:ieq}



\printglossary
\printglossary[type=\acronymtype, nonumberlist]
\clearpage
\end{document}
