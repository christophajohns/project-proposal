\documentclass[12pt,twoside,english]{article}
\usepackage[utf8]{inputenc}

%%%%%%%%%%%%%%%%%%%%%%%%%%%%%%%%%%%%%%%%%%%%%%%%%%%%%%%%%%%%%%%%%%%%%%%
%% template for II2202 proposal
%% original 2020.08.28
%% revised  
%%%%%%%%%%%%%%%%%%%%%%%%%%%%%%%%%%%%%%%%%%%%%%%%%%%%%%%%%%%%%%%%%%%%%%%
%

%%% Local Variables:
%%% mode: latex
%%% TeX-master: "."
%%% End:

%%TC:ignore
\usepackage[paper=a4paper,dvips,top=1.5cm,left=1.5cm,right=1.5cm,
    foot=1cm,bottom=1.5cm]{geometry}


\usepackage{todonotes}          %to provide inline and margin notes
%\usepackage[T1]{fontenc}
%%\usepackage{pslatex}
\renewcommand{\rmdefault}{ptm} 
\usepackage{mathptmx}
\usepackage[scaled=.90]{helvet}
\usepackage{courier}
%
\usepackage{bookmark}

\usepackage{fancyhdr}
\pagestyle{fancy}

%%----------------------------------------------------------------------------
%%   pcap2tex stuff
%%----------------------------------------------------------------------------
 %%\usepackage[dvipsnames*,svgnames]{xcolor} %% For extended colors
 %%\usepackage{tikz}  %% Already loaded
 %%\usetikzlibrary{arrows,decorations.pathmorphing,backgrounds,fit,positioning,calc,shapes}

%% \usepackage{pgfmath}	% --math engine
%%----------------------------------------------------------------------------
%% \usepackage[latin1]{inputenc}
\usepackage[utf8]{inputenc} % inputenc allows the user to input accented characters directly from the keyboard
\usepackage[english]{babel}
%% \usepackage{rotating}		 %% For text rotating
\usepackage{array}			 %% For table wrapping
\usepackage{graphicx}	                 %% Support for images
\usepackage{float}			 %% Suppor for more flexible floating box positioning
\usepackage{color}                       %% Support for colour 
\usepackage{mdwlist}
%% \usepackage{setspace}                 %% For fine-grained control over line spacing
%% \usepackage{listings}		 %% For source code listing
%% \usepackage{bytefield}                %% For packet drawings
\usepackage{tabularx}		         %% For simple table stretching
%%\usepackage{multirow}	                 %% Support for multirow colums in tables
\usepackage{dcolumn}	                 %% Support for decimal point alignment in tables
\usepackage{url}	                 %% Support for breaking URLs
\usepackage[perpage,para,symbol]{footmisc} %% use symbols to ``number'' footnotes and reset which symbol is used first on each page
\usepackage[binary-units=true]{siunitx} %% to be able to use binary units
\newcommand{\SIadj}[2]{\SI[number-unit-product={\text{-}}]{#1}{#2}}

%% \usepackage{pygmentize}           %% required to use minted -- see python-pygments - Pygments is a Syntax Highlighting Package written in Python
%% \usepackage{minted}		     %% For source code highlighting

%% \usepackage{hyperref}		
\usepackage[all]{hypcap}	 %% Prevents an issue related to hyperref and caption linking
%% setup hyperref to use the darkblue color on links
%% \hypersetup{colorlinks,breaklinks,
%%             linkcolor=darkblue,urlcolor=darkblue,
%%             anchorcolor=darkblue,citecolor=darkblue}

%% Some definitions of used colors
\definecolor{darkblue}{rgb}{0.0,0.0,0.3} %% define a color called darkblue
\definecolor{darkred}{rgb}{0.4,0.0,0.0}
\definecolor{red}{rgb}{0.7,0.0,0.0}
\definecolor{lightgrey}{rgb}{0.8,0.8,0.8} 
\definecolor{grey}{rgb}{0.6,0.6,0.6}
\definecolor{darkgrey}{rgb}{0.4,0.4,0.4}
%% Reduce hyphenation as much as possible
\hyphenpenalty=15000 
\tolerance=1000

%% useful redefinitions to use with tables
\newcommand{\rr}{\raggedright} %% raggedright command redefinition
\newcommand{\rl}{\raggedleft} %% raggedleft command redefinition
\newcommand{\tn}{\tabularnewline} %% tabularnewline command redefinition

%% definition of new command for bytefield package
\newcommand{\colorbitbox}[3]{%
	\rlap{\bitbox{#2}{\color{#1}\rule{\width}{\height}}}%
	\bitbox{#2}{#3}}

%% command to ease switching to red color text
\newcommand{\red}{\color{red}}
%%redefinition of paragraph command to insert a breakline after it
\makeatletter
\renewcommand\paragraph{\@startsection{paragraph}{4}{\z@}%
  {-3.25ex\@plus -1ex \@minus -.2ex}%
  {1.5ex \@plus .2ex}%
  {\normalfont\normalsize\bfseries}}
\makeatother

%%redefinition of subparagraph command to insert a breakline after it
\makeatletter
\renewcommand\subparagraph{\@startsection{subparagraph}{5}{\z@}%
  {-3.25ex\@plus -1ex \@minus -.2ex}%
  {1.5ex \@plus .2ex}%
  {\normalfont\normalsize\bfseries}}
\makeatother

\setcounter{tocdepth}{3}	%% 3 depth levels in TOC
\setcounter{secnumdepth}{5}
%% Acronyms
\usepackage[acronym, nopostdot]{glossaries}
\glsdisablehyper
\makeglossaries

\renewcommand{\headrulewidth}{0pt}
%%%%%%%%%%%%%%%%%%%%%%%%%%%%%%%%%%%%%%%%%%%%%%%%%%%%%%%%%%%%%%%%%%%%
%% End of preamble
%%%%%%%%%%%%%%%%%%%%%%%%%%%%%%%%%%%%%%%%%%%%%%%%%%%%%%%%%%%%%%%%%%%%

\newacronym{AR}{AR}{augmented reality}

\title{Effects of RealityKit AREnvironmentLighting on Immersion in Augmented Reality Scenes}
\author{
        \textsc{Stefano Formicola}
            \qquad
        \textsc{Christoph Albert Johns}
        \mbox{}\\
        \normalsize
            \texttt{formico}
        \textbar{}
            \texttt{cajohns}
        \normalsize
            \texttt{@kth.se}
}
\date{\today}


\lhead{II2202, Fall 2020, Period 1-2}
%% or \lhead{II2202, Fall 2020, Period 1}
\chead{Project proposal}
\rhead{\date{\today}}

\makeatletter
\let\ps@plain\ps@fancy 
\makeatother

\setlength{\headheight}{15pt}
\begin{document}

\maketitle


% \begin{abstract}
% \label{sec:abstract}

% Your abstract here.

% \end{abstract}
%%\clearpage

\selectlanguage{english}

\section{Allocation of responsibilities}
\label{sect:alloc_responsibilities}

Stefano Formicola is responsible for collecting data during the experiment, writing the research questions, method, results, and discussion, and presenting the final project.

Christoph Albert Johns is responsible for creating the test application, writing the abstract, introduction, and literature study, and presenting the project plan.


\section{Organization}
\label{sect:organization}

The research project will be conducted as a two-person project on the basis of previously published work in the domain of virtual lighting in augmented reality scenes and user-based studies.
First, the project's theoretical foundation will be studied in parallel with designing and preparing the test application.
Second, the experiment setup will be decided and prepared. In parallel the test application will be programmed and the basic project presentation will be prepared.
Third, the study participants will be continuously recruited and the required data will be collected and analyzed in parallel.
Fourth, another literature study will be conducted to relate the project's findings to existing work.
Finally, the project presentation will be completed and presented.


\section{Background}
\label{sect:background}
\todo[inline]{Describe the background for chosen area that is going to be investigated. Write a short description of the area that is going to be investigated. It is a brief description of the necessary background knowledge of the problem area and for carrying out the project. For example:}
This project builds on the idea of multicast file distribution described in RFC1235\cite{john_ioannidis_coherent_1991}. Clients report which blocks they are missing as a vector of bits, where missing blocks are indicated by a 1 bit. The length of each block is fixed; we will assume that it is 512 bytes for this project.



\section{Problem statement}
\label{sect:problem_statement}
\todo[inline]{Describe the problem(s) that have been found in the area described in the background. Describe the problem area (in detail). For example:}

The project will investigate how to avoid so-called “acknowledgement implosion” when distributing a file using multicast. If all of the nodes that successfully receive a packet were to acknowledge it, then the sender would receive a very large number of \glspl{ACK}, when it fact it is most interested in understanding which node did not receive the packet, hence to which node (or nodes) it should retransmit the packet.

\section{Problem}
\label{sect:problem}

\todo[inline]{State a clear and concise problem that is going to be investigated.  Answer the question What is the real problem? - What is the problem or value proposition addressed by the project? – Ideally one sentence that is very concrete. For example:}

Avoiding \gls{ACK} implosion is essential to enable multicast file distribution to scale to large numbers of receivers. What techniques can we use and how should they be used.

\section{Hypothesis}
\label{sect:hypothesis}
\todo[inline]{State a hypothesis that you think would be the outcome of your investigation. Answer the question: What is your hypothesis? (Note that the hypothesis must be measurable to be confirmed or falsified. Moreover not all projects have hypothesis.).}

Avoiding \gls{ACK} implosion can be performed by sending only \glspl{NACK}, rather than sending positive \glspl{ACK}.

\section{Purpose}
\label{sect:purpose}
\todo[inline]{Explain the purpose(s) of your project / investigation (the expected deliverables from the project). Answer the question: Why do this project? (purpose/effect, i.e. – the purpose can be to save environment but the goal is to build a robot that can pick up trash.) Why would you carry out the project? For example:}

The purpose is to present one or more techniques that can be used with multicast file distribution to prevent an \gls{ACK} implosion, thus fostering the use of multicast file distribution.

\section{Goal(s)}
\label{sect:goals}
\todo[inline]{Explain the goal(s), objective(s), and/or the result(s) of your investigation. What are the expected deliverables/outcomes from the project? For example:}

An analytic model showing the advantage of using only \glspl{NACK} versus positive \glspl{ACK} will be derived based on fitting curves to the experimental data for both forms of multicast file distribution. This model can be used to invalidate the hypothesis.

\section{Tasks}
\label{sect:tasks}
\todo[inline]{Describe the tasks and sub tasks that are necessary to complete the work. Grouped into a work breakdown structure. For example:}

A simple multicast file server and a corresponding client will be implemented. A number of instances of the client will be executed with different starting times, but starting within \SI{30}{\second} of each other. Our test file will be a file of sufficient size to require more than \SI{30}{\second} to transmit when transported using \gls{UDP} datagrams via a \SI{10}{Mbps} wired Ethernet. This file will contain a series of pseudo-random bytes. The total number of datagrams transmitted by all of the nodes will be compared when using positive \glspl{ACK} and when only generating \glspl{NACK}. The number of clients will vary from 1 to 1025, in increments of 8 clients.

\section{Method}
\label{sect:method}
\todo[inline]{Describe and explain the research methods that will be used for the project. What research method (or methods) will be used? 
Argue: why this is the most appropriate method or methods. For example:
}

The project will use the empirical method\cite{peter_bock_getting_2001}, as it is not clear to the authors that we can use an analytical method to produce an accurate model of the protocol in the available time period for this assignment (given the available resources).

\section{Milestone chart (time schedule)}
\label{sect:milestones}
\todo[inline]{Give a detailed schedule of how the project will be carried out. What is the project timeline and when will particularly meaningful points, referred to as milestones, be completed? What is the deliverable for each of these milestones? For example:}

The project will start on 25 August and end at 00:00 on 30 August. There will be the following milestones and deliverables:

\begin{description}
\item{26 August} working multicast server and client

\item{27 August} prepare testbed and perform experimental runs for each of the
  different numbers of clients to be completed (as per above) and stored in
  two comma separated files (ACK.csv and NACK.csv) which each line consisting
  of: number of clients, number of packets. Data will be collected using a
  separate computer running Wireshark. Analysis of the data and development of
  the analytic model will proceed in parallel with the data collection.

\item{28 August} model completed based upon using the statistical package R to do the curve fitting to the experimental data

\item{29 August} evaluation of model for additional data points to be check by new experimental runs to check that the model accurately produces the expected number of packets for the given number of clients for both \gls{ACK} and \gls{NACK} alternatives.

\item{Before 30 August} submit final report (the report will have been written in parallel with each of the above steps)
\end{description}

\bibliography{II2202-proposal}
%%\bibliographystyle{IEEEtran}
\bibliographystyle{myIEEEtran}

\appendix
\section{Optional appendix}

\todo[inline]{In this section you can additional information that may be relevant to your reader, but is not an answer to any of the above points. Note that the Appendix or Appendices are Optional.}

The total word count is XXX , with XXX words excluding title, authors, milestone/schedule, references, and words in this sentence, i.e., it is a word count from “Allocation of responsibilities” to end of “Method” section.
\todo[inline]{The value XXX does not include the instructions and the project plan should be handed in without instructions.}

\section{Acronyms}
\renewcommand{\glossarysection}[2][]{} %% skip the title
\printglossary[type=\acronymtype,nonumberlist]
\clearpage
\end{document}
