%%% Local Variables:
%%% mode: latex
%%% TeX-master: "."
%%% End:

%%TC:ignore
\usepackage[paper=a4paper,dvips,top=1.5cm,left=1.5cm,right=1.5cm,
    foot=1cm,bottom=1.5cm]{geometry}


\usepackage{todonotes}          %to provide inline and margin notes
%\usepackage[T1]{fontenc}
%%\usepackage{pslatex}
\renewcommand{\rmdefault}{ptm} 
\usepackage{mathptmx}
\usepackage[scaled=.90]{helvet}
\usepackage{courier}
%
\usepackage{bookmark}

\usepackage{fancyhdr}
\pagestyle{fancy}

%%----------------------------------------------------------------------------
%%   pcap2tex stuff
%%----------------------------------------------------------------------------
 %%\usepackage[dvipsnames*,svgnames]{xcolor} %% For extended colors
 %%\usepackage{tikz}  %% Already loaded
 %%\usetikzlibrary{arrows,decorations.pathmorphing,backgrounds,fit,positioning,calc,shapes}

%% \usepackage{pgfmath}	% --math engine
%%----------------------------------------------------------------------------
%% \usepackage[latin1]{inputenc}
\usepackage[utf8]{inputenc} % inputenc allows the user to input accented characters directly from the keyboard
\usepackage[english]{babel}
%% \usepackage{rotating}		 %% For text rotating
\usepackage{array}			 %% For table wrapping
\usepackage{graphicx}	                 %% Support for images
\usepackage{float}			 %% Suppor for more flexible floating box positioning
\usepackage{color}                       %% Support for colour 
\usepackage{mdwlist}
%% \usepackage{setspace}                 %% For fine-grained control over line spacing
%% \usepackage{listings}		 %% For source code listing
%% \usepackage{bytefield}                %% For packet drawings
\usepackage{tabularx}		         %% For simple table stretching
%%\usepackage{multirow}	                 %% Support for multirow colums in tables
\usepackage{dcolumn}	                 %% Support for decimal point alignment in tables
\usepackage{url}	                 %% Support for breaking URLs
\usepackage[perpage,para,symbol]{footmisc} %% use symbols to ``number'' footnotes and reset which symbol is used first on each page
\usepackage[binary-units=true]{siunitx} %% to be able to use binary units
\newcommand{\SIadj}[2]{\SI[number-unit-product={\text{-}}]{#1}{#2}}

%% \usepackage{pygmentize}           %% required to use minted -- see python-pygments - Pygments is a Syntax Highlighting Package written in Python
%% \usepackage{minted}		     %% For source code highlighting

%% \usepackage{hyperref}		
\usepackage[all]{hypcap}	 %% Prevents an issue related to hyperref and caption linking
%% setup hyperref to use the darkblue color on links
%% \hypersetup{colorlinks,breaklinks,
%%             linkcolor=darkblue,urlcolor=darkblue,
%%             anchorcolor=darkblue,citecolor=darkblue}

%% Some definitions of used colors
\definecolor{darkblue}{rgb}{0.0,0.0,0.3} %% define a color called darkblue
\definecolor{darkred}{rgb}{0.4,0.0,0.0}
\definecolor{red}{rgb}{0.7,0.0,0.0}
\definecolor{lightgrey}{rgb}{0.8,0.8,0.8} 
\definecolor{grey}{rgb}{0.6,0.6,0.6}
\definecolor{darkgrey}{rgb}{0.4,0.4,0.4}
%% Reduce hyphenation as much as possible
\hyphenpenalty=15000 
\tolerance=1000

%% useful redefinitions to use with tables
\newcommand{\rr}{\raggedright} %% raggedright command redefinition
\newcommand{\rl}{\raggedleft} %% raggedleft command redefinition
\newcommand{\tn}{\tabularnewline} %% tabularnewline command redefinition

%% definition of new command for bytefield package
\newcommand{\colorbitbox}[3]{%
	\rlap{\bitbox{#2}{\color{#1}\rule{\width}{\height}}}%
	\bitbox{#2}{#3}}

%% command to ease switching to red color text
\newcommand{\red}{\color{red}}
%%redefinition of paragraph command to insert a breakline after it
\makeatletter
\renewcommand\paragraph{\@startsection{paragraph}{4}{\z@}%
  {-3.25ex\@plus -1ex \@minus -.2ex}%
  {1.5ex \@plus .2ex}%
  {\normalfont\normalsize\bfseries}}
\makeatother

%%redefinition of subparagraph command to insert a breakline after it
\makeatletter
\renewcommand\subparagraph{\@startsection{subparagraph}{5}{\z@}%
  {-3.25ex\@plus -1ex \@minus -.2ex}%
  {1.5ex \@plus .2ex}%
  {\normalfont\normalsize\bfseries}}
\makeatother

\setcounter{tocdepth}{3}	%% 3 depth levels in TOC
\setcounter{secnumdepth}{5}
%% Acronyms
\usepackage[acronym, nopostdot]{glossaries}
\glsdisablehyper
\makeglossaries

\renewcommand{\headrulewidth}{0pt}
%%%%%%%%%%%%%%%%%%%%%%%%%%%%%%%%%%%%%%%%%%%%%%%%%%%%%%%%%%%%%%%%%%%%
%% End of preamble
%%%%%%%%%%%%%%%%%%%%%%%%%%%%%%%%%%%%%%%%%%%%%%%%%%%%%%%%%%%%%%%%%%%%
